\documentclass[a4paper,12pt]{article}
\usepackage{fullpage,times}
\usepackage{verbatim}  % for comments
% \usepackage{euro}
% \EUR~10.10

%\input{Figs}
\input mm_macros.tex

\DeclareMathAlphabet{\mathbit}{OT1}{cmr}{bx}{it}
\def\vv#1{\ensuremath{\mathbit{\bf #1}}}


\begin{document}
\thispagestyle{empty}
\pagestyle{plain}
\centerline{\Large\bf Comments on the specs02.pdf}
\bigskip

\centerline{\large\bf {\em Marek}}
\bigskip

\centerline{\sc \today}

\bigskip
\bigskip

In order to assure consistency between the problem formulation and the
corresponding model, I would consider replacement of the obviously incorrect
inequality~(9) by:
\btla{88.}
\inum Equation defining the amounts of each of the fuels as a function of
the activities of the applied technologies:
	\be
	\sum_{i \in I} a_{ji} \cdot ACT_i^t = x_j^t, \quad j \in J,\; t \in T
	\ee
	where:
	\btlbs
	\item $j$ denotes fuel type, $J = \{gasoline, diesel\}$,
	\item $a_{ji}$ relates the amount of $j$-th fuel produced by the unit of
		$i$-th ACT,
	\item $x_j^t$ stands for the amount of $j$-th fuel produced jointly by all
		considered technologies at period~$t$.
	\etls
\inum Adding the supply-demand constraint, specification of which depends on the
	chosen definition of demand. Here we can consider one of the following two options:
	\btlas{88.}
	\inums If the demand is given for each fuel type, i.e., as $d_j^t$, then:
		\be\label{eq:dem1}
		x_j^t \ge d_j^t, \quad j \in J,\; t \in T.
		\ee
	\inums If the demand is given for a linear aggregation of $d_j^t$, e.g.,
		by coefficients $\alpha_j$ conforming to:
		\be
		0 \le \alpha_j \le 1,\; \forall j \in J; \quad \sum_{j \in J} \alpha_j = 1.
		\ee
		Thus, the demand is given for a {\em virtual} (i.e., not actually existing)
		fuel:
		\be
			d^t = \sum_{j \in J} \alpha_j \cdot d_j^t, \quad t \in T.
		\ee
		In such a case instead constraint~(\ref{eq:dem1}) one shall add constraint:
		\be
		\sum_{j \in J} \alpha_j \cdot x_j^t \ge d^t, \quad t \in T.
		\ee
	\etls
\etl

\end{document}
